\documentclass{article}
\usepackage[top=2cm, left=2cm, right=2cm, bottom=3cm]{geometry}
%\usepackage{lipsum} % untuk \lipsum

\title{Rencana Proyek, Mata Kuliah Pemrograman Fungsional: \\
\textit{Human Task Scheduler} dengan Metaheuristik Algoritma Genetika di Haskell}
\author{Tim/Kelompok: satu orang (individu) \\
Bisma Rohpanca Joyosumarto (2106635581)}
\date{Oktober 2024}

% agar tidak ada indentasi dan tidak perlu \noindent
\setlength{\parindent}{0pt}

% terkait link/hyperlink
\usepackage{hyperref}
\hypersetup{
    colorlinks=true,
    linkcolor=blue,
    filecolor=magenta,
    urlcolor=blue % bisa juga misalnya cyan
}

\begin{document}

\maketitle

\textbf{Topik:}
\begin{itemize}
    \item \textbf{Utama:} Metaheuristik "algoritma genetika" \textit{(genetic algorithm)} dengan pendekatan fungsional
    \begin{itemize}
        \item Memiliki unsur \textit{random}, yang kemudian harus diurus secara \textit{purely functional}
    \end{itemize}
    \item Masalah optimasi berupa penjadwalan pengerjaan tugas (berkaitan dengan manajemen waktu) untuk manusia, sebagai contoh penerapan metaheuristik \textit{(metaheuristics)}
\end{itemize}

Note: saya sudah cukup familiar dengan metaheuristik dan masalah optimasi di Departemen Matematika, hanya saja belum pernah menyelam perihal \textit{job scheduling} atau semacamnya. \\

\textbf{Teknologi yang digunakan:} Haskell.

\begin{itemize}
    \item Untuk \textit{backend}: Haskell
    \item Untuk \textit{frontend}: Masih sedang memilih antara beberapa teknologi \textit{frontend} untuk Haskell. Saat ini sedang memilih antara \textbf{Miso} (\url{https://haskell-miso.org/}) atau \textbf{Reflex} (\url{https://reflex-frp.org/})
\end{itemize}

\textbf{URL-Repository:} \url{https://github.com/BismaBRJ/human-task-scheduler-genetic-algo-hs}

\section*{Target per pekan}

\subsection*{Target Progress 1 (31 Oktober 2024)}

Menyusun \textit{greedy task scheduler} sederhana, belum melibatkan metaheuristik dan belum memiliki \textit{frontend}. Masih melalui \textit{command line} atau terminal. Tujuannya adalah menyiapkan struktur data dan sistem I/O \textit{(input/output)}, juga memulai proses perumusan masalah optimasi.

\begin{enumerate}
    \item Menyusun struktur data tugas \textit{(task)} termasuk menyimpan \textit{deadline} dan jadwal pengerjaannya, struktur data jadwal (mungkin sebagai sekumpulan tugas), struktur data jam kosong (yang nantinya bisa dijadwalkan pengerjaan tugas di jam segitu), dan mungkin struktur data lainnya kalau diperlukan.
    \item Juga menyiapkan beberapa fungsi yang bisa dengan mudah meng-\textit{extract} data tertentu, misalnya memperoleh semua \textit{deadline}, atau memperoleh tugas dengan \textit{deadline} terawal.

    \item Menyiapkan kendala \textit{(constraints)} paling sederhana, yaitu bahwa tugas sebaiknya/harus dikerjakan sebelum \textit{deadline}, dan dua tugas tidak bisa dikerjakan secara bersamaan, dalam bentuk fungsi.
    
    Mungkin fungsi akan \textit{return} tipe data \textbf{Bool}, atau bisa juga berupa nilai (berapa satuan waktu sebelum \textit{deadline}; semakin besar, semakin baik).

    Persiapan kendala adalah dalam rangka memformulasikan masalah optimasi yang perlu diselesaikan.

    \item Mempelajari dan menyusun algoritma \textit{greedy} untuk penjadwalan, secara \textit{purely functional}, hingga bisa digunakan sebagai fungsi.
    
    Mungkin masih sangat sederhana, intinya semua tugas terjadwal se-awal mungkin, dan yang dikerjakan terlebih dahulu adalah yang paling dekat \textit{deadline}-nya, belum ada \textit{break} antar jadwal pengerjaan tugas.

    Mungkin saja melibatkan studi literatur.
\end{enumerate}

\textbf{Konsep pemrograman fungsional yang terlibat:} konsep dasar seperti rekursi, \textit{pattern matching}, tipe data, serba-serbi list (terutama cons dan/atau append, mungkin \textit{list comprehension} juga).

\subsection*{Target Progress 2 (7 November 2024)}

Mulai membuat \textit{frontend} atau UI \textit{(user interface)} sederhana, agar tugas bisa diinput dan hasil penjadwalan bisa ditampilkan. Juga memperbaiki sedikit, algoritma \textit{greedy} yang sudah dibuat agar juga mengadakan \textit{break} antar jadwal pengerjaan dua tugas.

\begin{enumerate}
    \item Memilih antara Miso atau Reflex, lalu mempelajari \textit{framework} yang dipilih dan menggunakannya untuk membuat UI sederhana untuk input tugas dan menampilkan jadwal.
    
    Juga mempelajari bagaimana \textit{frontend} bisa disambung dengan \textit{backend}.

    \item Memodifikasi sedikit algoritma \textit{greedy} yang dibuat agar juga berusaha mengadakan \textit{break} antar dua jadwal pengerjaan tugas, tetap mematuhi kendala bahwa tiap tugas sebaiknya/harus diselesaikan sebelum \textit{deadline}.
    
    Mungkin saja melibatkan studi literatur.
\end{enumerate}

\textbf{Konsep pemrograman fungsional yang terlibat:} sudah mulai bermain dengan dunia luar yang sifatnya \textit{impure}, sehingga mungkin sudah melibatkan konsep seperti \textbf{Monad}. Juga tetap melibatkan konsep dasar, dalam rangka memodifikasi algoritma \textit{greedy} tersebut.

\subsection*{Target Progress 3 (14 November 2024)}

Mulai mengimplementasikan algoritma genetika. Juga menambahkan fitur di \textit{frontend} agar bisa input parameter untuk algoritma genetika.

\begin{enumerate}
    \item Merumuskan masalah optimasi untuk permasalahan ini, meliputi fungsi objektif \textit{(objective function)}, juga disebut \textit{fitness function}, yang ingin dimaksimumkan, dan juga meliputi semua kendala yang sudah dipertimbangkan sejauh ini.
    
    Formulasi masalah optimasi bisa saja melibatkan parameter yang kemudian bisa ditentukan, biasanya untuk menentukan prioritas, misalnya antara seberapa awal tugas dikerjakan sebelum \textit{deadline} dan seberapa lama \textit{break} antar pengerjaan dua tugas.
    
    Mungkin saja melibatkan studi literatur.

    \item Mempelajari dan mengimplementasikan algoritma genetika, terutama tahap \textit{crossover}, \textit{mutation}, dan \textit{selection}.
    
    Algoritma genetika aslinya bersifat iteratif, sehingga di sini harus diimplementasikan secara rekursif.

    Ketiga tahap tersebut memiliki unsur \textit{random}, yang juga menjadi tantangan tersendiri dalam \textit{purely functional programming}.
    
    Juga mengimplementasikan fungsi-fungsi untuk tahap-tahap tersebut, khusus untuk konteks permasalahan yang sedang dihadapi.

    Mungkin saja melibatkan studi literatur.

    \item Tambahkan fitur input parameter di \textit{frontend}, baik untuk menentukan prioritas di masalah optimasi, maupun untuk algoritma genetika.
    
    Tentunya, sambung kembali \textit{backend} terbaru ke \textit{frontend}.

    Mungkin, buat juga tambahan agar user bisa memilih antara algoritma \textit{greedy} yang telah dibuat sebelumnya atau algoritma genetika.
\end{enumerate}

\textbf{Konsep pemrograman fungsional yang terlibat:} kembali melibatkan konsep dasar, terutama untuk implementasi algoritma genetika, yang aslinya bersifat iteratif. Juga mungkin saja melibatkan \textit{higher-order function}, karena algoritma genetika sebenarnya cukup \textit{general}, kita bisa mendefinisikan fungsi \textit{crossover} dsb khusus kasus ini di luar inti algoritma genetika itu sendiri. Mungkin melibatkan \textbf{Monad} untuk urusan \textit{random}. Mungkin juga melibatkan \textit{lazy evaluation}, karena dalam tahap \textit{selection} perlu dipilih sekian dari populasi yang sangat banyak (walaupun tidak tak hingga karena masalah ini masih diskret).

\subsection*{Target Progress 4 (21 November 2024)}

Menambahkan fitur/kendala baru: ketergantungan tugas atau urutan tugas \textit{(task precedence)}, yaitu bahwa bisa saja ada tugas yang harus dikerjakan terlebih dahulu sebelum tugas lain. \\[0.5em]
Ini mempengaruhi perumusan masalah optimasi, sehingga mempengaruhi \textit{backend}, meliputi strutkur data (sekarang tugas bisa memiliki prasyarat) dan algoritma genetika (terutama untuk pendefinisian fungsi objektif yang baru). Juga mempengaruhi \textit{frontend}, untuk urusan input data dan tampilan.

\begin{enumerate}
    \item Memodifikasi struktur data tugas agar tiap tugas bisa memiliki prasyarat.
    
    Juga menambahkan fungsi-fungsi tambahan untuk ekstraksi data.
    
    Memodifikasi struktur data lainnya apabila diperlukan.

    \item Membuat fungsi-fungsi tambahan yang memeriksa kendala terkait \textit{precedence}. Misalnya, fungsi yang memeriksa apakah penjadwalan tertentu memenuhi kendala \textit{precedence} tersebut.

    \item Membuat struktur data graf, lebih tepatnya \textit{dependency graph}, untuk menyimpan hubungan ketergantungan antar tugas.
    
    Juga membuat fungsi yang memanfaatkan struktur data graf ini untuk memeriksa, dari segi ketergantungan tugas, apakah penjadwalan mungkin sama sekali, yaitu apakah graf tersebut merupakan DAG \textit{(directed acyclic graph)}. Fungsi ini kemudian perlu dilibatkan dalam pencarian solusi, sebelum menggunakan algoritma genetika.
    
    \item Memodifikasi perumusan masalah optimasi dengan menambahkan unsur \textit{precedence}, yaitu dengan menambahkan kendala. Parameter bisa saja bertambah.
    
    \item Memodifikasi \textit{fitness function} untuk penggunaan algoritma genetika agar juga mempertimbangkan \textit{precedence}.
    
    \item Menambahkan fitur input ketergantungan tugas di \textit{frontend}, dan melapor apabila pencarian solusi tidak mungkin (misalnya karena \textit{dependency graph} yang terbentuk tidak berupa DAG).
\end{enumerate}

\textbf{Konsep pemrograman fungsional yang terlibat:} kembali melibatkan konsep dasar, terutama seputar struktur data. Namun, kalau perlu, struktur data DAG bisa dicoba dijadikan \textit{instance} dari berbagai \textit{type class}, misalnya \textbf{Traversable} atau bahkan \textbf{Applicative Functor}.

\subsection*{Target Progress 5 (28 November 2024)}

Menambahkan \textit{error handling} dan/atau \textit{form validation}, juga fitur \textit{save} dan \textit{load} data.

\begin{enumerate}
    \item Mulai menerapkan \textit{error handling} dan/atau \textit{form validation}, misalnya dengan memanfaatkan tipe \textbf{Maybe}.
    
    Ini bisa saja melibatkan modifikasi urusan tipe di keseluruhan kode, atau bisa juga di fungsi-fungsi tertentu saja, terutama yang dekat dengan \textit{frontend}.

    \item Selain itu, juga membuat fitur \textit{save} dan \textit{load} data, baik data tugas maupun data hasil penjadwalan.
    
    \item Kalau sempat, tambahkan fitur di \textit{frontend} untuk juga menampilkan \textit{dependency graph}. Visualisasi seperti itu bisa membantu pengguna.
        
    Mungkin saja harus selektif atau setidaknya diatur kembali, bagian apa saja dari tiap tugas yang bisa ditampilkan, agar mudah dibaca dan tidak begitu memakan tempat.
\end{enumerate}

\textbf{Konsep pemrograman fungsional yang terlibat:} seputar tipe \textbf{Maybe}, sehingga bisa saja sampai melibatkan konsep \textbf{Applicative Functor}. Juga memerlukan \textbf{Monad} yang berkaitan dengan \textit{save} dan \textit{load} data, kemungkinan \textbf{IO Monad}.

\subsection*{Target Progress 6 (5 Desember 2024)}

\textit{Finishing up}, baik dari segi \textit{backend} (mempercepat algoritma genetika) maupun \textit{frontend}, dan mulai eksplor \textit{deployment}.

\begin{enumerate}
    \item Mulai mempelajari dan melakukan eksplorasi seputar \textit{deployment}, dan mencoba men-\textit{deploy} keseluruhan aplikasi sedemikian sehingga bisa digunakan oleh orang lain melalui suatu \textit{website}.
    
    Sebelumnya, \textit{website} tersebut perlu dipilih, untuk urusan \textit{hosting}.

    \item Menulis panduan atau README untuk proyek, melibatkan cara \textit{running}, cara menggunakan, bahkan cara \textit{deploy} sendiri.
    
    \item Mencoba mengimplementasikan paralelisasi \textit{(parallelism)} di algoritma genetika. Di tiap iterasi, banyak anggota populasi perlu melalui tahapan \textit{crossover} dan \textit{mutation}; pengerjaan kedua tahap tersebut bisa saja dibagi-bagi ke sejumlah \textit{core}. Setelah selesai, hasilnya disatukan kembali ke dalam satu \textit{list} (atau semacamnya), barulah bisa melalui tahap \textit{selection} untuk melanjutkan ke iterasi selanjutnya.
    
    Mungkin juga mencoba melakukan \textit{fine-tuning}, yaitu mencoba mencari parameter yang cukup optimal untuk algoritma genetika.

    Mungkin saja melibatkan studi literatur, terutama untuk pemilihan parameter.
    
    \item Mempelajari dan eksplorasi tentang FaaS \textit{(function as a service)}, mencoba melakukan penyesuaian pada kode \textit{backend} hingga bisa di-\textit{deploy} ke FaaS.
\end{enumerate}

\textbf{Konsep pemrograman fungsional yang terlibat:} melibatkan \textbf{Monad}, misalnya untuk mencoba paralelisasi, bisa juga urusan \textit{deployment}. Tentunya melibatkan juga konsep FaaS.

\end{document}